\subsubsection{概要}
Python\cite{python}はWindows,Linux/Unix,Mac OS X などの主要なオペレーティングシステムおよびJavaや.NETなどの仮想環境でも動作するインタプリタ形式の,対話的な,オブジェクト指向プログラミング言語である.
この言語には,モジュール,例外,動的な型付け,超高水準の動的データ型,およびクラスが取り入れられている.
Pythonはオブジェクト指向プログラミングを超えて,手続き型プログラミングや関数型プログラミングなど複数のプログラミングパラダイムをサポートしている.
また,多くのシステムコールやライブラリだけでなく,様々なウィンドウシステムへのインターフェースがあり,C\cite{Clang}やC++\cite{cplusplus}で拡張することもできる.

\subsubsection{Django}
Django\cite{Django}は,Pythonで実装された無料オープンソースのWebアプリケーションフレームワークの一つである.
Djangoが作られた時の目的として,複雑なデータベース主体のウェブサイトを簡単に構築するというものがある.
これを実現するため,Djangoではコンポーネントの再利用性,素早い開発の原則に力を入れている.
このような流れから,Djangoには以下のような特徴がある.

\begin{description}
    \item[・高速な動作]\mbox{}\\
        Djangoには標準で分散型のキャッシュシステムであるmemcached\cite{memcached}が備え付けられており,キャッシュ機能が強力である.
    \item[・フルスタック・フレームワーク]\mbox{}\\
        Djangoには,Webアプリケーションの実装に必要な,ユーザ認証,管理画面,RSSフィードなどの機能があらかじめ含まれている.
    \item[・セキュリティ的に安全な設計]\mbox{}\\
        Djangoでは,デフォルトでパスワードなどはハッシュ化しデータベースに格納する.
        また,SQLインジェクション,クロスサイトスクリプティング,クロスサイトリクエストフォージェリなどの多くの脆弱性についても保護を有効にしている.
    \item[・自由に選択できるプラットフォーム]\mbox{}\\
        Djangoは,そのすべてがPythonから作成されている.これにより,PythonがLinux,Windows,MacOS Xなどで実行できるようにDjangoも多くのプラットフォームで動作可能である.
\end{description}


