\begin{center}
{\bf \Large 概要}
\end{center}

総務省が令和元年に37182人に行った調査によると,インターネットの利用率は,その約9割にまで増加している\cite{soumu}.その理由の1つにSNSの普及が挙げられる.
SNSに関して平成27年にみずほ情報総研株式会社が1178人に行った調査研究\cite{mizuho}によると,SNS上でトラブルの経験があると回答した割合は15\%程であった.
トラブルの内容は,自分自身の発言を他人が異なる意味で受け取ってしまう,自分の意志とは関係なく個人情報などが第三者に公開されてしまうなどである.
このようなトラブルを避けるために情報倫理教育は有効な手段の一つである\cite{moraru}.

さらに,新しい情報倫理の問題に対応するために,個人の継続的な学習が要求される\cite{fluency}.
しかし,従来の書籍による学習では,新しい問題への速やかな対応が難しいため,学習の継続が難しい場合がある.
他方,eラーニング学習は,学習コンテンツの更新が容易なため,学習の継続が期待できる\cite{chieru}.

そこで本研究では,持続的な学習環境を提供することを目的に,情報倫理教育におけるe ラーニングのためのプラットフォーム(以下,本プラットフォーム)を開発する.
まず,本稿ではコンテンツの作成,統計情報の確認,外部アプリケーションの導入を補佐する機能を開発した.
本プラットフォームを用いることで,情報倫理に関するコンテンツを web 上で管理,提供でき,持続的なコンテンツの提供が行える.
これらのコンテンツを用いて学習することにより,トラブルの減少やリテラシーの向上が期待できる.

