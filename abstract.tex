\begin{center}
{\bf \Large 概要}
\end{center} 
総務省が令和元年に37182人に行った調査によると,インターネット利用者の割合は9割に迫るところまで増加している\cite{soumu}.インターネットの利用率が増加する中,インターネットの利用理由としてあげられるものの内の一つとしてSNSが存在する.
このSNSにおいて平成27年にみずほ情報総研株式会社が1178 人に行った調査研究2)によると,SNS上でトラブルの経験があると回答した割合は15%程であった.トラブルの内容は,自分自身の発言が他人に異なる意味で受け取られてしまう,自分の意志とは関係なく個人情報などが他人に公開されてしまう等である.このようなトラブルを避けるために情報倫理教育が必要である.情報倫理教育の内容として個人情報の保護,人権侵害,コンピュータ犯罪等がある.これらの教育は特にインターネットの利用において重要である3).
そこで本研究では,情報倫理教育に関する学習と教育を支援することを目的に,情報倫理に関するコンテンツを提供可能とするプラットフォーム(以下,本プラットフォーム)を開発する.
本プラットフォームを用いることで,情報倫理に関するコンテンツをweb上で管理,提供できる.これにより情報倫理を学ぶ際,コンテンツを用いて学習することにより,トラブルの減少やリテラシーの向上がの理解がを期待できる.
