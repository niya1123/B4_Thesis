関連研究として,上田氏らの「倫倫姫プロジェクト-学人連携Moodleによる多言語情報倫理eラーニング-」\cite{rinri}がある.

倫倫姫プロジェクトでは大学の情報倫理教育における以下3つの問題を解決している.

\begin{enumerate}[(i)]
    \item 標準化と可視化がなされていない \label{i}
    \item 留学生への教育が困難 \label{ii}
    \item 持続可能性が低い \label{iii}
\end{enumerate}

(\ref{i})を解決するために,倫倫姫ではサンプル規程集「A3301 教育テキスト作成ガイドライン(一般利用者向け)」に準拠することで,内容を標準化した.
また,受講履歴を閲覧でき受講者の学習状況の可視化も実現した.

(\ref{ii})を解決するために,倫倫姫では英語,中国語,韓国版を作成しており,各言語圏の文化の違いも考慮しコンテンツを作成することで解決した.

(\ref{iii})を解決するために,倫倫姫ではSCORMが規定するeラーニングコンテンツパッケージを利用した.
これは,パッケージ構造とリソースを記述するマニフェストファイル(imsmanifest.xml)とそれから参照される物理ファイル(HTML,swfなど)をファイル単位で修正可能なため継続的な改訂を可能としている.

本研究でも(\ref{i}),(\ref{iii})については同様に本プラットフォームで解決している.
(\ref{i})に関しては,コンテンツ提供機能により,教材提供者間でコンテンツを共有できることから解決している.
(\ref{iii})に関しては,本プラットフォームではコンテンツはWebアプリケーション上で管理しており,コンテンツの修正もWeb上から行えることから解決している.

一方,本プラットフォームにおいては,コンテンツに関してはマークダウン形式で記入可能なことから,過去のコンテンツを本プラットフォーム上に移行することが簡単である.
また,コンテナ管理機能により,過去に作成したアプリケーションに関しても本プラットフォーム上に移行することが簡単である.

